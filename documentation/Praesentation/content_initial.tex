\titlepageframe % Specific command 

\begin{tframe}{Introduction}
\begin{itemize}
\item Beamer is a \LaTeX{} class that allows you to create presentations
\item Beamer2Thesis is a Beamer package that allows you to create a presentation for your thesis
\begin{itemize}
\item with specific dedicated commands
\item it uses TorinoTh, a Beamer Theme
\end{itemize}
\end{itemize}
\end{tframe}

\begin{tframe}{TorinoTh theme}
\begin{itemize}
\item TorinoTh (TorinoThesis) is a theme which is based on Torino and extend it
\begin{itemize}
\item Torino is a pretty theme for Beamer realized by Marco Barisione
\item \href{http://blog.barisione.org/2007-09/torino-a-pretty-theme-for-latex-beamer/}{http://blog.barisione.org/2007-09/torino-a-pretty-theme-for-latex-beamer/}
\end{itemize}
\item Names are similar because I attend, as him, the Politecnico of Torino, but I want to emphasize the fact that TorinoTh is a theme that can be used only to realize a thesis
\end{itemize}
\end{tframe}

\begin{tframe}{TorinoTh theme}
\begin{itemize}
\item The theme consist of:
\begin{itemize}
\item \emph{beamercolorthemetorinoth.sty} defines colors and allows you to choose among three possible options: blue (default), green or red
\item \emph{beamerfontthemetorinoth.sty} defines fonts used
\item \emph{beamerinnerthemetorinoth.sty} defines the title page and items
\item \emph{beamerouterthemetorinoth.sty} defines headers and footers
\item \emph{beamerthemeTorinoTh.sty} include all definitions
\end{itemize}
\end{itemize}
\end{tframe}

\begin{tframe}{Installation}
Beamer2Thesis is distributed under:
\begin{itemize}
\item TeX Live
\item MiKTeX
\end{itemize}
You can use, respectively, the Package Wizard for MiK\TeX\, (\emph{Start/MiKTeX/2.9/}) and the TeX Live Manager for \TeX Live to search and install the theme.
Personally, I suggest you to use \TeX Live: it works for Linux, Mac and Windows. Actually, the installation under Linux is a bit complicated, but I have realized a short guide that may help you
\begin{itemize}
\item \href{http://claudiofiandrino.altervista.org/LaTeX/texlive_on_ubuntu_guide.pdf}{download the guide here}
\end{itemize}

\end{tframe}

\begin{tframe}{Installation (II)}
\label{slide-guide-installation}
As an alternative, Beamer2Thesis can be downloaded from my personal page as a zip file
\begin{itemize}
\item \href{http://claudiofiandrino.altervista.org/latex\_projects.html}{http://claudiofiandrino.altervista.org/latex\_projects.html}
\end{itemize}
or in the official page:
\begin{itemize}
\item \href{http://cfiandra.github.com/Beamer2Thesis/}{http://cfiandra.github.com/Beamer2Thesis/}
\end{itemize}

It can be installed with the standard procedure used to install a common package: I suggest you to read a short guide I have written
\begin{itemize}
\item \href{http://claudiofiandrino.altervista.org/LaTeX/package_installation.pdf}{download the guide here}
\end{itemize}
\end{tframe}

\begin{tframe}{The guides}
\begin{itemize}
\item Next slides will present all features avaiable
\item As examples in which different options are applied, is possible to see the guides:
\begin{itemize}
\item beamer2thesis.pdf is the standard english guide which uses standard options
\item beamer2thesis\_ita.pdf is the italian guide with green colors
\end{itemize}
\end{itemize}
\end{tframe}

\begin{frame}[t,fragile]{How to read the guides}
\begin{itemize}
\item All guides show options in general; to have a look for specific configurations, read each guide because in each one is reported its own configuration state
\item Every time something is declared to be \emph{default}, it is possible to omit it from the configuration phase
\item Every time an option is enabled by setting it with \emph{true}, to disable it you can use \emph{false}; for example:
\begin{verbatim}
secondcandidate=false 
secondcandidate=true 
\end{verbatim}
\end{itemize}
\end{frame}

\begin{frame}[t,fragile]{The configuration phase}
\begin{itemize}
\item It is the first thing you have to declare in the document
\item The general code is \verb!\usetheme[.. options ..]{TorinoTh}!
\item An example is:
\begin{verbatim}
\documentclass{beamer}
\usetheme[language=english,
          titlepagelogo=logopolito,
          bullet=circle,
          pageofpages=of,
          titleline=true,
          color=blue
          ]{TorinoTh}
\end{verbatim}
\end{itemize}
\end{frame}

\begin{tframe}{Some general options}
\begin{itemize}
\item The \highlight{pageofpages} option defines the string between the current
      page number and the total page count
  \begin{itemize}
  \item the default is \emph{of}
  \end{itemize}
\item If the \highlight{titleline} option is set to \emph{true}, a horizontal line
      is drawn below the title
  \begin{itemize}
  \item the default is \emph{true}; use \emph{false} to disable
  \end{itemize}
\item The \highlight{notshowauthor} option set to \emph{true} allows you to not show the name of the author in the footer
\begin{itemize}
\item the default is \emph{false}
\end{itemize}
\item The \highlight{titlepagelogo} is the name of the principal logo: it must be a .jpg, .pdf, .png picture
\begin{itemize}
\item to include the logo of your University, follow the procedure explained in the following slide
\end{itemize}
\end{itemize}
\end{tframe}

\begin{tframe}{How insert a new logo}
There are several ways to do it (for people highly capable in \LaTeX\, this is not a problem), but I suggest this method:
\begin{itemize}
\item download from my page the .zip file and extract it
\item copy your logo into the directory of the package
\item install the package in your personal tree following the guide reported in slide \ref{slide-guide-installation}
\end{itemize}
\label{slide-rule-installation}
\end{tframe}

\begin{frame}[t,fragile]{Other options: avaiable bullets}
\begin{itemize}
\item The \highlight{bullet} option can be used to choose the symbol used in the bullet lists
  \begin{itemize}
  \item \verb!square!: a filled square
        ({\usebeamercolor[fg]{item}\tiny\raise0.2ex\hbox{$\blacksquare$}}) for
        first and third level items, an empty square
        ({\usebeamercolor[fg]{item}\tiny\raise0.2ex\hbox{$\square$}}) for
        second level items
  \item \verb!diamond!: a filled diamond
        ({\usebeamercolor[fg]{item}\tiny\raise0.2ex\hbox{$\blacklozenge$}}) for
        first and third level items, an empty diamond
        ({\usebeamercolor[fg]{item}\tiny\raise0.2ex\hbox{$\lozenge$}}) for
        second level items
  \item \verb!triangle!: a filled triangle
        ({\usebeamercolor[fg]{item}\tiny\raise0.2ex\hbox{$\blacktriangleright$}}) for
        first and third level items, an empty triangle
        ({\usebeamercolor[fg]{item}\tiny\raise0.2ex\hbox{$\vartriangleright$}}) for
        second level items
  \item \verb!circle!: a filled circle ({\usebeamercolor[fg]{item}$\bullet$})
        for first and third level items, an empty circle
        ({\usebeamercolor[fg]{item}$\circ$}) for second level items
  \item The default value is \verb!circle!
  \end{itemize}
\end{itemize}
\end{frame}

\begin{frame}[t,fragile]{Languages}
\begin{itemize}
\item All languages can be supported, but the two main ones are:
\begin{itemize}
\item english
\item italian 
\end{itemize}
\item The choice of one of the main languages implies that in the titlepage, date and labels (Supervisor, Candidate, Relatore, Candidato) are shown with the proper language in an automatic way
\item To set the italian language, for example, use in the configuration phase:
\verb!language=italian!; the name should be the one used by the package babel or by \verb!\setmainfont! with \XeLaTeX
\item If the language selected is not one of the two main languages, then labels in the titlepage should be introduced by the user (see the example in the next frame)
\end{itemize}
\end{frame}

\begin{frame}[t,fragile]{Languages (II)}
\begin{itemize}
\item Example with spanish language:
\begin{verbatim}
\usetheme[language=spanish,...]{TorinoTh}
\setrellabel{Relator Tesis}
\setcandidatelabel{Candidato}
\setassistentsupervisorlabel{Co Tesis}
\setsubject{Tesis}
\end{verbatim}
\item Commands illustrated are mandatory when \highlight{not using} a main language
\item If you have already set a language and you change, it may happen that, the first time you compile, this error occurs: \begin{flushleft}
\highlight{! Package babel Error: You haven't loaded the option -language- yet}
\end{flushleft}
do not be afraid and compile a second time: it will work!!
\end{itemize}
\end{frame}

\begin{frame}[t,fragile]{Coding}
To avoid forcing an user to use the \verb!utf8x! coding, this release fix the bug by introducing the \highlight{coding} option; possible choices you can exploit are:
\begin{itemize}
\item \verb!coding=utf8x! (default)
\item \verb!coding=utf8! 
\item \verb!coding=latin1! 
\end{itemize}
An important advise: the program does not check which string you put in input; it is your matter select the right coding to satisfy requeriments of your system.
\end{frame}

\begin{frame}[t,fragile]{Second logo}
\begin{itemize}
\item If, for some reasons, someone needs a second logo (a thesis performed in another institute for example) an option allows you to put it in the title page
\item When \highlight{secondlogo} is set to \emph{true}, you have to use the command \verb!\titlepagesecondlogo{name-logo}!: otherwise an error occurs
\item As the main logo, the second logo must be a .jpg, .pdf, .png picture and you can insert it following the same rules explained in slide \ref{slide-rule-installation}
\end{itemize}
\end{frame}

\begin{frame}[fragile]{Third logo}
\begin{itemize}
\item Eventually, if you need a third logo you can exploit the possibility of insert it by setting the option \highlight{thirdlogo} to \emph{true}
\begin{itemize}
\item the default is \emph{false}
\end{itemize}
\item You have to insert the picture as described for the second logo and use the command \verb!\titlepagethirdlogo{name-logo}!
to put the logo in the title page
\item Of course, you can use this option if, and only if, the \highlight{secondlogo} is set to \emph{true}
\item When there are three logos please use, as reference for the dimensions, the picture \alert{logopolito}: in this way they will be aligned
\end{itemize}
\end{frame}

\begin{frame}[t,fragile]{Second candidate}
\begin{itemize}
\item It is possible that there are two candidates: the package manage this fact easily
\begin{itemize}
\item the first candidate is also the author
\item the second candidate can be inserted with the command \verb!\secondcandidate{name-surname}! when the option \highlight{secondcandidate} is set to \emph{true}
\end{itemize}
\item Of course, when there are two candidates the label \emph{Candidate} becomes \emph{Candidates} and \emph{Candidato} become \emph{Candidati}
\item With two candidates, the footer changes and the author is not shown automatically (the reason is simply: show two authors plus the title is too much long, making the footer too big)
\end{itemize}
\end{frame}

\begin{frame}[t,fragile]{Supervisor and Assistant Supervisor}\label{secondrel}
\begin{itemize}
\item To insert the supervisor you just have to use the command \verb!\rel{name-surname}!
\item There is also the possibility of report the Assistant supervisor:
\begin{itemize}
\item set the option \highlight{assistantsupervisor} to \emph{true} (default is \emph{false})
\item use the command \verb!\assistantsupervisor{name-surname}!
\end{itemize}
\item Labels are set accordingly to the language used
\end{itemize}
\end{frame}

\begin{frame}[t,fragile]{Second Supervisor and Assistant Supervisor}
There is also the possibility of insert more than one supervisor and assistant supervisor:
\begin{itemize}
\item set the options:
\begin{itemize}
\item \highlight{secondsupervisor} to true (default is false);
\item \highlight{secondassistantsupervisor} to true (default is false);
\end{itemize}
\item name can be inserted thanks to:
\begin{itemize}
\item \verb!\secondsupervisor! command;
\item \verb!\secondassistantsupervisor! command; this one can be exploited just when the \highlight{assistantsupervisor} option is set to true;
\end{itemize}
\item as usual, labels are set accordingly to the language used
\end{itemize}
\end{frame}

\begin{frame}[t,fragile]{Advantages and Disadvantages}
Sometimes it is useful highlight advantages and disadvantages of a given argument: instead of list them by using the standard bullet, there is the possibility of exploit two new environments (\emph{adv} and \emph{disadv}). Usage:
\begin{columns}
\begin{column}{0.3\paperwidth}
\begin{verbatim}
\begin{adv}
\item 
\end{adv}
\end{verbatim}
\end{column}
\begin{column}{0.3\paperwidth}
\begin{verbatim}
\begin{disadv}
\item 
\end{disadv}
\end{verbatim}
\end{column}
\end{columns}
\bigskip
In the following slide there is an example.
\end{frame}

\begin{tframe}{Why use Beamer2Thesis}
Advantages:
\begin{adv}
\item Simply to install
\item Easy to customize
\item Possibility to exploit several features
\end{adv}
Disadvantages:
\begin{disadv}
\item Difficulty with long titles
\item If you find some others, please contact me 
\end{disadv}

\end{tframe}

\begin{frame}[t,fragile]{Finally: colors}
\begin{itemize}
\item There are three possible choices:
\begin{itemize}
\item blue
\item green
\item red
\end{itemize}
\item When the color is chosen setting the option \highlight{color} to one of the list above, consequently headers, footers, title page, bullet and highlightings are set accordingly
\item For example: \verb!color=green!
\end{itemize}
\end{frame}

\begin{tframe}{\XeLaTeX}
Thanks to a suggestion and the precious help of Nicola Tuveri, Beamer2Thesis supports \XeTeX\, and \XeLaTeX\, automatically.
You can choose your favourite font to further customize the presentation. I report some examples:\\
\highlight{Uncomment following lines of code if you use \XeLaTeX!}
%%% commented by Karol Kozioł
%\fontspec[Ligatures={Common, Historical}]{Linux Libertine O Italic}
%\fontsize{12pt}{18pt}\selectfont This is quite strange! 
%\fontspec{TeX Gyre Pagella}
%\selectfont{Also this is strange}\\
%\fontspec[ SizeFeatures={
%{Size={-10}, Font=TeX Gyre Bonum Italic, Color=AA0000},
%{Size={10-14}, Color=00AA00},
%{Size={14-}, Color=0000FA}} ]{TeX Gyre Chorus}
%\selectfont{How to customize fonts?}\par
%\begin{itemize}
%\item {\LARGE Word}
%\item Word
%\item {\tiny World}
%\end{itemize}
\end{tframe}

\begin{frame}[t,fragile]{\XeLaTeX\,: code}
To realize the examples reported in the previous slide, the code is:
\scriptsize{
\begin{verbatim}
\fontspec[Ligatures={Common, Historical}]{Linux Libertine O Italic}
\fontsize{12pt}{18pt}\selectfont This is quite strange!
\fontspec{TeX Gyre Pagella}
\selectfont{Also this is strange}\\
\fontspec{TeX Gyre Pagella}
\selectfont{How to customize fonts?}\par
\fontspec[ SizeFeatures={
{Size={-10}, Font=TeX Gyre Bonum Italic, Color=AA0000},
{Size={10-14}, Color=00AA00},
{Size={14-}, Color=0000FA}}]{TeX Gyre Chorus}
\begin{itemize}
  \item {\LARGE Word}
  \item Word
  \item {\tiny World}
\end{itemize}
\end{verbatim}
}
\end{frame}

\begin{tframe}{Block}
Beamer allows to use the \emph{block} environment: it is very useful in some applications. For example:
\begin{block}<1->{Why use Beamer2Thesis? Advantages}
\begin{adv}
\item Simply to install
\item Easy to customize
\item Possibility to exploit several features
\end{adv}
\end{block}
\begin{block}<2->{Why use Beamer2Thesis? Disadvantages}
\begin{disadv}
\item Difficulty with long titles
\item If you find some others, please contact me 
\end{disadv}
\end{block}
\end{tframe}

\begin{frame}[t,fragile]{Block: code}
The previous slide has been realized as:
\small{\begin{verbatim}
\begin{block}<1->{Why use Beamer2Thesis? Advantages}
\begin{adv}
\item Simply to install
\item Easy to customize
\item Possibility to exploit several features
\end{adv}
\end{block}
\begin{block}<2->{Why use Beamer2Thesis? Disadvantages}
\begin{disadv}
\item Difficulty with long titles
\item If you find some others, please contact me 
\end{disadv}
\end{block}
\end{verbatim}}
\end{frame}

\begin{frame}[t,fragile]{Block: code (II)}
More in general, Beamer provide three \emph{block} environments:
\begin{itemize}
\item \highlight{block}
\item \highlight{alertblock}
\item \highlight{exampleblock}
\end{itemize}
To have more details, not only on this argument, I suggest to read the \href{http://mirrors.ctan.org/macros/latex/contrib/beamer/doc/beameruserguide.pdf}{beameruserguide}.
\end{frame}
