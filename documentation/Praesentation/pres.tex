\documentclass{beamer}
\usetheme{metropolis}           % Use metropolis theme
%\usetheme{CambridgeUS}
\title{Kolloquium zur Bachelorarbeit}
\date{07. M{\"a}rz 2018}

\institute{Universit{\"a}t Trier}

\usepackage{listings}
\lstset{
  numbers=left,
  stepnumber=1,    
  firstnumber=0,
  numberfirstline=true
}

\setbeamertemplate{footline}
{
  \leavevmode%
  \hbox{%
  \begin{beamercolorbox}[wd=.4\paperwidth,ht=2.25ex,dp=1ex,center]{author in head/foot}%
    \usebeamerfont{author in head/foot}Benedikt L{\"u}ken-Winkels
  \end{beamercolorbox}%
  \begin{beamercolorbox}[wd=.6\paperwidth,ht=2.25ex,dp=1ex,center]{title in head/foot}%
    \usebeamerfont{title in head/foot}\insertshorttitle\hspace*{3em}
    \insertframenumber{} / \inserttotalframenumber\hspace*{1ex}
  \end{beamercolorbox}}%
  \vskip0pt%
}


%Mathematikumgebungen
\usepackage{amsmath,amsthm,amssymb}

\usepackage{threeparttable}
\usepackage{ngerman}
\usepackage[utf8]{inputenc}
\graphicspath{{img/}}

\definecolor{amber}{rgb}{1.0, 0.75, 0.0}
\setbeamercolor{block title}{use=text,
    fg=amber,
    bg=gray}
\setbeamercolor{block body}{use={block title , text},
    fg=text.fg,
    bg=lightgray}
    
\begin{document}
\author{%
\begin{tabular}{l l} 
Referent:   & Benedikt L{\"u}ken-Winkels \\[1ex] 
Pr{\"u}fer:  & Prof. Dr. Henning Fernau\\
             & Prof. Dr.  Stefan N{\"a}her
\end{tabular}}
	\maketitle
	\section{Knotenüberdeckungsproblem}
	\begin{frame}{Knotenüberdeckungsproblem - Definition}
\begin{block}{Knotenüberdeckung}
EINGABE: $\ Graph\ G=(V,E),\ positive\ Integer\ k\leq |V|$\\
AUSGABE: $\ S\subseteq V\ mit\ |S|\leq k,$ sodass\ jede\ Kante\ aus\ E\ einen\ Endpunkt\ in\ S\ hat.
\end{block}			
		
	\end{frame}
	\section{Graphreduktion}
	\begin{frame}{}
	\end{frame}
  
	\section{Einfache Reduktionsregeln}
	\begin{frame}{}
	\end{frame}
	
	\section{Kronenregel}
    \begin{frame}{}
	\end{frame}
	
	\section{Nemhauser-Trotter-Regel}
	\begin{frame}{}
	\end{frame}
	
	\section{Laufzeit und erwartete Reduktion}
	\begin{frame}{}
	\end{frame}
	\subsection{Test}
	\section{Anwendung}
	\begin{frame}{}
	\begin{table}[htbp]
\caption{Anwendung kombinierter Reduktionsregeln\label{tab:kombination}}
\vspace*{1em}
\centering

\bgroup
\def\arraystretch{1.3}%  1 is the default, change whatever you need
\tiny
\begin{tabular}[c]{l|l|l|l|l}

	
	\multicolumn{1}{c|}{\textbf{Kombination}} &
	\multicolumn{1}{c|}{\textbf{Anwendungen$_{1}$}} &
	\multicolumn{1}{c|}{\textbf{Anwendungen$_{2}$}} &
	\multicolumn{1}{c|}{\textbf{Anwendungen$_{3}$}} & 
	\multicolumn{1}{c}{\textbf{Reduktion}} \\
	\hline

	K - G$_{1}$ & 3.63 & 4.3 & - &331.8\\
	G$_{1}$ - K & 4.37 & 3.22 & - &331.17\\
	K - NT & 0.8 & 0.38 & - & 68.28 \\
	NT - K & 0.45 & 0.56 & - & 68.6\\
	G$_{1}$ - NT & 1.33 & 0.017 & - & 99.87\\
	NT - G$_{1}$ & 0.28 & 1.13 & - & 99.87\\
	K  - G$_{1}$ - NT & 3.61 & 4.29 & 0.11 & 334.67 \\
	K - NT - G$_{1}$ & 3.6 & 0.87 & 3.39 & 334.83 \\
	G$_{1}$ - NT - K & 4.36 & 0.12 & 3.2 & 334.17 \\
	G$_{1}$ - K - NT & 3.61 & 3.2 & 0.65 & 334.16 \\
	NT - K - G$_{1}$ & 0.39 & 3.44 & 4.03 & 335.2 \\
	NT - G$_{1}$ - K & 0.91 & 3.42 & 3.2 & 334.16 \\

	
\end{tabular}

\egroup

\end{table}
	\end{frame}
	
	\section{Implementierung}
	\begin{frame}{}
	\end{frame}
	
	\section{Fazit}
	\begin{frame}{}
	\end{frame}
  
\end{document}