\documentclass[a4paper]{article}
\usepackage{ngerman}
\usepackage{amsmath}
\usepackage{listings}
\usepackage{hyperref}
\usepackage[backend=bibtex]{biblatex} 
\lstset{
  numbers=left,
  stepnumber=1,    
  firstnumber=0,
  numberfirstline=true
}
\bibliography{Sources} 
\usepackage[utf8]{inputenc}
\usepackage[T1]{fontenc}



\begin{document}
\title{Bachelorabeit\\ Untersuchung von Datenreduktionsregeln beim  Kontenüberdeckungsproblem}
\author{Benedikt Lüken-Winkels}
\maketitle
\tableofcontents
\newpage
\begin{abstract}
Was ist Vertex Cover und warum sollte man es reduzieren; Es gibt verschiedene Algorithmen, die in Polinomialzeit einen Problemkern erstellen.
\end{abstract}
\section{Einleitung}
\begin{itemize}
\item Einfluss von Reduktionsregeln auf die Problemgröße
\end{itemize}


\section{Knotenüberdeckungsproblem}
\begin{itemize}
\item Woher kommt die Komplexität?
\item Was macht eine schwere Instanz aus?
\item Wie sieht eine schwere Instanz aus?
\item Wo findet das Knotenüberdeckungsproblem Anwendung?
\end{itemize}

\section{Graphreduktion}
\begin{itemize}
\item Effekt von Graphreduktionsalgorithmen auf die Problemkomplexität
\item Bewertungskriterien für einen GRalgorithmus
	\begin{itemize}
	\item Laufzeit (Parametrisierung)
	\item Erwartete Reduktion/Wie oft wird die Regel angewandt
	\item Ressourcenverbrauch
	\item Wie gut ist das Ergebnis im Vergleich zu anderen Algorithmen?	
	\end{itemize}
\item Wie funktionieren die GRA in Kombination?
\item Wie sehen Graphen aus, auf die keine Regel anwendbar ist?
\item Wie sehen Graphen aus, auf die genau eine Regel anwendbar ist?
\item Welche Regeln werden untersucht?
\end{itemize}
\subsection{Parametrisierte Algorithmen}
Kleiner Exkurs
\begin{itemize}
\item Wie funktioniert Parametrisierung?
\item Vorteile von FPA
\end{itemize}
\subsection{Einfache Reduktionsregeln}
\subsubsection{Grad 0, Grad 1}
Selbsterlklärend

\subsection{Buss}

\subsection{Nemhauser/Trotter}

\subsubsection{Theorie}

\subsubsection{Implementierung/Umsetzung}

\subsubsection{Ergebnisse}


\subsection{Kronenregel}

\subsubsection{Theorie}

\subsubsection{Implementierung/Umsetzung}

\subsubsection{Ergebnisse}

\section{Analyse}

\subsection{Vergleich}

\newpage
\printbibliography

\end{document}
