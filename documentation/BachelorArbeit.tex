\documentclass[a4paper]{article}
\usepackage{ngerman}
\usepackage{amsmath}
\usepackage{listings}
\usepackage{hyperref}
\usepackage[backend=bibtex]{biblatex} 
\lstset{
  numbers=left,
  stepnumber=1,    
  firstnumber=0,
  numberfirstline=true
}
\bibliography{Sources} 
\usepackage[utf8]{inputenc}
\usepackage[T1]{fontenc}

\begin{document}
\title{Bachelorabeit\\ Untersuchung von Datenreduktionsregeln beim  Kontenüberdeckungsproblem}
\author{Benedikt Lüken-Winkels}
\maketitle
\tableofcontents
\newpage
\begin{abstract}
-Was ist Vertex Cover und warum sollte man es reduzieren; \cite{paper:1} Es gibt verschiedene Algorithmen, die in Polinomialzeit einen Problemkern erstellen.
\end{abstract}
\section{Einleitung}



\newpage
\printbibliography

\end{document}
