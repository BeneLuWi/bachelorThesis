%% Einleitung.tex
%% $Id: einleitung.tex 61 2012-05-03 13:58:03Z bless $
%%

\chapter{Einleitung}
\label{ch:Einleitung}
%% ==============================
Die Einleitung besteht aus der Motivation, der Problemstellung, der Zielsetzung und einem erster Überblick über den Aufbau der Arbeit.

%% ==============================
\section{Motivation}
%% ==============================
\label{ch:Einleitung:sec:Motivation}

\begin{itemize}
\item Definition
\item Wo findet das Knotenüberdeckungsproblem Anwendung?
\item Woher kommt die Komplexität?
\item Was macht eine schwere Instanz aus?
\item Wie sieht eine schwere Instanz aus?
\end{itemize}
Das \emph{Knotenüberdeckungsproblem}, oder englisch Vertex Cover, ist ein nachgewiesen NP-vollständiges Problem \cite{intract}. Um das Problem zu erklären, betrachten wir ein Netz von Haushalten, bei dem wir die Möglichkeit haben, in jedem Haushalt einen Stromgenerator zu platzieren, sodass durch alle mit diesem Haus verbundenen Leitungen Strom fließt. Ziel ist, ein stabiles Stromnetz zu schaffen, bei dem jede Leitung an mindestens eine Stromquelle angeschlossen ist. Nun ergibt sich das Problem, dass die Kosten für Stromgeneratoren erheblich sind und daher nur maximal \emph{k} Geräte angeschafft werden können. Es gilt also, aus den \emph{n} Häusern \emph{k} oder weniger auszuwählen, sodass jede Leitung von einem der Häuser in der Auswahl versorgt wird. 
\begin{figure}[htb]
\centering
  	{\includegraphics[width=.5\textwidth]{vertexcoverBsp.pdf}}
	\caption{Graph eines Stromnetzes \label{fig:vc}}
\centering
\end{figure}
In Abbildung \ref{fig:vc}\footnote{Icons in der Graphik von https://www.flaticon.com/}   sieht man ein Beispiel für eine solche Probleminstanz. Die Haushalte sind jeweils mit einem Buchstaben gekennzeichnet; jede Linie, beziehungsweise Kante eine Leitung zum Nachbarhaus. Hier ist es möglich eine Lösung für \emph{k} = 4 zu finden, also 4 Häuser mit Generatoren auszustatten, sodass alle Leitungen versorgt sind. Es ist leicht zu sehen, dass es keine Lösung mit weniger Häusern gibt. Um 4 Häuser, die das Problem lösen zu finden, betrachtet man nun jede Leitung und entscheidet, welches der beiden Häuser aufgenommen werden soll, da mindestens eins der beiden Häuser in der Lösungsmenge ist. Werden alle $2^{8}$ Möglichkeiten ausprobiert, finden sich die Lösung, dass wenn die Haushalte F, B, A und C oder F, B, A und D mit einem Generator ausgestattet werden, durch alle Leitungen Strom fließt. Für eine Problemgröße wie sie hier geschildert wird, bietet die \emph{Brute-Force}-Methode eine Lösung. Für eine Häusermenge \emph{n} = 2000 mit entsprechend vielen Leitungen bedeutet das $1.148 \cdot 10^{602}$ mögliche Kombinationen. 




Es ist leicht zu sehen, dass sich das Stromnetzproblem, ersetzt man die Häuser durch Knoten eines Graphen und die Leitungen durch dessen Kanten,  zum Knotenüberdeckungsproblem transformieren lässt, welches sich folgendermaßen definiert\cite{trees}:
\begin{align*}
EINGABE: &\ Graph\ G=(V,E),\ positive\ Integer\ k\leq |V|\\
AUSGABE: &\ S\subseteq V\ mit\ |S|\leq k,\\
&\ sodass\ jede\ Kante\ aus\ E\ einen\ Endpunkt\ in\ S\ hat.
\end{align*}
Ein nichtdeterministischer Algorithmus kann jede vermeintliche Lösungskombination aus Knoten, beziehzungsweise Haushalten in Polynomialzeit testen \cite{intract}, da lediglich überprüft werden muss, ob die Größe der Lösungsmenge den Wert von \emph{k} überschreitet und ob alle Kanten abgedeckt (Leitungen versorgt) sind.
 Allerdings würde ein solcher Algrithmus für große Eingaben in der Realität nicht in absehbarer Zeit terminieren, wie beispielhaft in Tabelle \ref{tab:exponential} zu sehen ist.
 
 \begin{table}[htb]
\caption{Exponentielle Laufzeit \label{tab:exponential}}
\vspace*{1em}
\centering

\bgroup
\def\arraystretch{1.3}%  1 is the default, change whatever you need

\begin{threeparttable}

\begin{tabular}[c]{ l | l }
	
	\multicolumn{1}{c|}{\textbf{Eingabe}} & 
	\multicolumn{1}{c}{\textbf{Laufzeit}} \\ 
	
	\hline

	$2^{1}$& 0.00002s\\
	$2^{10}$& 0.02s\\
	$2^{100}$& ca. $ 2 \cdot 10^{14} $ Jahrtausende \\
	
\end{tabular}

\begin{tablenotes}\footnotesize
\item EINFÜGEN
\end{tablenotes}

\end{threeparttable}

\egroup

\end{table}
Es scheint also sinnvoll zu sein, die Eingabe möglichst klein zu halten und zu \emph{reduzieren}, indem einfache Teile des Graphen durch Vorverarbeitung (englisch \emph{Preprocessing}) entfernt werden und gegebenenfalls in die Lösungsmenge aufgenommen werden. Hierfür bieten sich einige einfache Beobachtungen, die über einen Graphen getroffen werden können an. Auf das Stromnetzproblem einem Netz mit maximal \emph{k} Generatoren bezogen bedeuten diese: 
\begin{enumerate}
\item Häuser, die isoliert stehen und keine Leitungen haben, können aus dem Netz entfernt werden
\item Bei Häuser, die genau eine Leitung, beziehungsweise genau ein Nachbarhaus haben, bekommt entsprechende Nachbarhaus automatisch einen Stromgenerator (in Zukunft Grad$_{1}$-Regel)
\item Häuser, die $k+1$ Leitungen zu anderen Häusern haben, erhalten einen Generator, da sonst alle Nachbarnhäuser versorgt würden und somit mehr Generatoren, als verfügbar verteilt werden müssten (Buss-Regel)
\end{enumerate}
Angewandt an einer Probleminstanz bedeutet das Einsetzen einer der Regeln, dass die entsprechenden Häuser und Leitungen von einer weiterführenden Betrachtung ausgeschlossen sind, da so eine
....Definition Reduktion.....
Um das Problem weiter zu vereinfachen stehen allerdings auch weiteaus komplexere Reduktionsregeln zur Verfügung. In dieser Arbeit werden das Nemhauser-Trotter-Theorem, beziehungsweise die dazugehörige Reduktionsregel und die Kronenregel betrachet.


Eine schnelle Laufzeit, um eine Knotenüberdeckung in einem gegebenen Graphen \emph{G}, Parameter \emph{k} und einer Knontenmenge \emph{n} zu finden wird von Chen et al \cite{paper:4} mit $O(kn + 1.271^{k}k^{2})$ vorgeschlagen; die schnellste bekannte Methode, alle Knotenüberdeckungen eines Graphen zu zählen von Mölle et al \cite{paper:5} mit einer Laufzeit von $O(1.3803^{k})$.

%% ==============================
\section{Problemstellung}
%% ==============================
\label{ch:Einleitung:sec:Problemstellung}

\begin{itemize}
\item Effekt von Graphreduktionsalgorithmen auf die Problemkomplexität
\end{itemize}

%% ==============================
\section{Zielsetzung}
%% ==============================
\label{ch:Einleitung:sec:Zielsetzung}

\begin{itemize}
\item Kategorisierung der Regeln?
\item Bewertungskriterien für einen GRalgorithmus
	\begin{itemize}
	\item Laufzeit (Parametrisierung)
	\item Erwartete Reduktion/Wie oft wird die Regel angewandt
	\item Ressourcenverbrauch
	\item Wie gut ist das Ergebnis im Vergleich zu anderen Algorithmen?	
	\end{itemize}
\item Wie funktionieren die GRA in Kombination?
\item Wie sehen Graphen aus, auf die keine Regel anwendbar ist?
\item Wie sehen Graphen aus, auf die genau eine Regel anwendbar ist?
\item Welche Regeln werden untersucht?
\end{itemize}


%% ==============================
\section{Gliederung/Aufbau der Arbeit}
%% ==============================
\label{ch:Einleitung:sec:Gliederung}

Was enthalten die weiteren Kapitel? Wie ist die Arbeit aufgebaut? Welche Methodik wird verfolgt?


%%% Local Variables: 
%%% mode: latex
%%% TeX-master: "thesis"
%%% End: 
