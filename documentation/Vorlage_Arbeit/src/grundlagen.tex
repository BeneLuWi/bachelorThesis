%% grundlagen.tex
%% $Id: grundlagen.tex 61 2012-05-03 13:58:03Z bless $
%%

\chapter{Grundlagen}
\label{ch:Grundlagen}
%% ==============================
Beschreibung der verschiedenen Reduktionsregeln und wie funktionieren. Mit graphischen Beispielen und Pseudocode?


%% ==============================
\section{Knotenüberdeckung}
%% ==============================
\label{ch:Grundlagen:sec:Knotenüberdeckung}

Eine schnelle Laufzeit, um eine Knotenüberdeckung in einem gegebenen Graphen \emph{G}, Parameter \emph{k} und einer Knontenmenge \emph{n} zu finden wird von Chen et al \cite{paper:4} mit $O(kn + 1.271^{k}k^{2})$ vorgeschlagen; die schnellste bekannte Methode, alle Knotenüberdeckungen eines Graphen zu zählen von Mölle et al \cite{paper:5} mit einer Laufzeit von $O(1.3803^{k})$.

%% ==============================
\section{Einfache Reduktionsregeln}
%% ==============================
\label{ch:Grundlagen:sec:Einfache Reduktionsregeln}
Aus einfachen Beobachtungen über den Problemgraphen lassen sich für einen Graphen $G=(V,K)$ und der natürlichen Zahl $k$ einige Reduktionsregelen ableiten.
\begin{enumerate}
\item Knoten, die isoliert stehen und keine Kanten haben, können aus dem Graphen entfernt werden, da sie nicht zu einer minimalen Knotenüberdeckung gehören können (Grad$_{0}$-Regel)
\item Bei Knoten, die genau eine Kante, beziehungsweise genau einen Nachbarknoten haben, wird der  entsprechende Nachbarknoten automatisch in die Lösungsmenge aufgenommen, da einer der beiden Knoten in der Knotenüberdeckung vorkommt und der Nachbarknoten mindestens den Grad 1 hat, also möglicherweise mehr Kanten abdeckt (Grad$_{1}$-Regel)
\item Knoten mit $k+1$ Kanten, werden der Knotenüberdeckung hinzugefügt, da sonst alle Nachbarknoten aufgenommen werden und somit mehr, als $k$ Knoten in der Lösungsmenge wären (Buss-Regel)
\end{enumerate}
Angewandt an eine Probleminstanz bedeutet das Einsetzen einer der Regeln, dass die entsprechenden Knoten und Kanten von einer weiterführenden Betrachtung ausgeschlossen sind. Buss-Regel und Grad$_{1}$-Regel unterscheiden sich von der Grad$_{0}$-Regel dahingehend, dass das Auslösen einer dieser Regeln auch den Parameter $k$ verringert, da ein Knoten in die Überdeckung aufgenommen wird.
 Das Beispiel in Abbildung \ref{fig:vc} kann durch die wiederholte Anwendung der Regeln sogar gelöst werden:
\begin{enumerate}
\item Es stehen 4 Generatoren zur Verfügung, allerdings hat Haus B 5 ausgehende Leitungen $\Rightarrow$ Buss-Regel: B erhält einen Generator und dessen Leitungen haben Strom.
\item Drei disjunkte Mengen von Häusern (\{E, G, F\}, \{D, C\} und \{A, H\}) bleiben übrig, wo jeweils einmal die Grad$_{1}$-Regel angewandt werden kann. Die 3 restlichen Generatoren werden verteilt, wobei sich bei jeder der Häusergruppen mehrere Möglichkeiten bieten.
\item Es steht kein Generator mehr zur Verfügung und aus der Restmenge {E, G, F} bleibt ein Haus ohne stromlose Leitungen zurück $\Rightarrow$ das Haus wird aus der Betrachtung entfernt und das Problem ist gelöst.
\end{enumerate}
Nach der wiederholten Anwendung der Buss-Regel haben alle Häuser mit mehr als \emph{k} Leitungen einen Generator und sind von der weiteren Betrachtung ausgeschlossen. Jeder übrige Haushalt kann also maximal \emph{k} stromlose Leitungen haben. Wenn nun noch mehr als $k^{2}$ Leitungen zu versorgen sind, kann es keine Lösung dieser Probleminstanz geben, da lediglich \emph{k} Stromgeneratoren zur Verfügung stehen und, verteilt auf die Häuser, somit nur maximal $k^{2}$ Leitungen unter Strom setzen können. Es bleiben höchstens $2 k^{2}$ Häuser ohne Generator (mit mindestens einer Leitung), bei denen noch nicht alle Leitungen Strom haben übrig\cite{param}, da jede der Leitungen an 2 Häusern angeschlossen ist. Wird nun die Grad$_{1}$-Regel angewandt bis sie nicht mehr greift, sind höchstens $k^{2} + k$ Häuser übrig, da jedes dieser Häuser jetzt zwischen 2 und \emph{k} stromlose Leitungen hat.


Um ein Haus unter den \emph{n} Häusern mit mehr als \emph{k} Leitungen zu finden, werden im schlimmsten Fall $k \cdot n$ Schritte benötigt, da für jedes Haus alle Leitungen gezählt werden müssen. Falls alle Häuser \emph{k} Leitungen haben, wird die Buss-Regel nicht ausgelöst. Ansonsten wird das gefundene Haus mit all seinen Leitungen aus dem Netz entfernt, was wiederum $d \geq k$(wobei $d$ der größte Grad unter den Knoten aus $G$ ist) Schritte benötigt. Die Grad$_{1}$-Regel benötigt im schlimmsten Fall $2 \cdot n$ Schritte um das Häusernetz einmal zu durchlaufen, da die Betrachtung eines Hauses abgebrochen werden kann, sobald es mehr als eine Leitung hat, das es dann für diese Regel uninteressant ist. Wird ein Haus gefunden, kann dessen Nachbar und alle von diesem ausgehenden Leitungen entfernt werden, also $d$. Wird die Buss-Regel wurde vor der Grad$_{1}$-Regel angewandt, bedeutet das, dass maximal \emph{k} Schritte gebraucht werden. Um alle Häuser ohne Leitungen zu entfernen, muss lediglich für jedes Haus überprüft werden, ob es Leitungen hat, also \emph{n} Schritte.




In \emph{O}-Notation ergibt sich daraus eine Laufzeitabschätzung eines Durchlaufs von
\begin{align}
\text{Worst Case}:\ & kn + p + p + 1\\
\Rightarrow\ & O(kn + p)\\
\Rightarrow\ & O(kn)
\end{align}

Um größere Instanzen weiter zu vereinfachen stehen allerdings auch weiteaus komplexere Reduktionsregeln zur Verfügung. In dieser Arbeit werden das Nemhauser-Trotter-Theorem, beziehungsweise die dazugehörige Reduktionsregel und die Kronenregel betrachet.




%% ==============================
\section{Nemhauser-Trotter Reduktionsregeln}
%% ==============================
\label{ch:Grundlagen:sec:Nemhauser-Trotter Reduktionsregeln}
$\textit{Für einen Graphen}\ G=(V,E)\textit{ können zwei disjunkte Mengen}$\\ $\ C_{0}\ und\ V_{0} \textit{ gefunden werden, sodass}$
\begin{enumerate}
\item $C_{0}$ \textit{ in einer minimalen Knotenüberdeckung} \\
\textit{von G enthalten ist,}
\item \textit{der Teilgraph }$G[V_{0}]$ \textit{eine Knotenüberdeckung}\\
\textit{der Größe} $\leq |V_{0}| / 2$ \textit{ hat,}
\item \textit{und} $VC(G) = VC(G[V_{0}])\cup C_{0}$ \textit{ gilt.}
\end{enumerate}

\begin{lstlisting}[mathescape = true, basicstyle=\ttfamily]
$G = (V, E), n:= |V|, m:=|E|, d:= maximaler\ Grad\ der\ Knoten\ aus\ G$
Bipartiden Graphen erstellen $B = (V, V', E')$ 
  mit $E':= \{\{x,y'\}, \{x', y\} | \{x,y\} \in E\}$ 
Maximum Matching $M$ von $B$ bestimmen 
$C_{B}:= VC(B)$ 
$C_{0}:= \{x \in V\ |\ x \in C_{B}\ und\ x' \in C_{B} \}$ 
$V_{0}:= \{x \in V\ |\ entweder\ x \in C_{B}\ oder\ x' \in C_{B} \}$ 
\end{lstlisting}
%% ==============================
\section{Kronenregel}
%% ==============================
\label{ch:Grundlagen:sec:Kronenregel}

\textit{Für einen Graphen} $G=(V,E)$ \textit{besteht eine Krone aus} $H \subseteq V\ und\ I \subseteq V\ mit\ H \cap I = \emptyset,\ sodass$ 
\begin{enumerate}
\item $H = N(I),$ 
\item $\forall v, w \in I\ gilt\ (vw) \notin E\ und$
\item \textit{die Kanten zwischen H und I enthalten ein Matching in dem alle Knoten aus H enthalten sind.}
\end{enumerate}


\begin{lstlisting}[mathescape=true, escapechar = !,basicstyle=\ttfamily]
$G = (V, E), n:= |V|, m:=|E|, d:= maximaler\ Grad\ der\ Knoten\ aus\ G$
$M_{1}$ := Maximal Matching von $G$
  $M_{1} := \emptyset$
  $\forall e \in$ E:
    $M_{1} = M_{1} \cup e$
    Entferne $e$ und $N(e)$ aus der weiteren Betrachtung
$O$ := nicht gepaarte Knoten in $M_{1}$
$M_{2}$ := Maximum Matching von $B = (O, N(O), \{ uv| u \in O \wedge v \in N(O)\}) $
$I$  := nicht gepaarte Knoten aus $O$ in $M_{2}$
$I'  := \emptyset$
while $I' \neq I$
  $I' := I$
  $H:= N(I)$
  $I := I \cup \{\forall u \in O|\exists v\in H\ (uv \in M_{2})\}$
Entferne $N(I)$ aus $G$
\end{lstlisting}


%%% Local Variables: 
%%% mode: latex
%%% TeX-master: "thesis"
%%% End: 
