%% zusammenf.tex
%% $Id: zusammenf.tex 61 2012-05-03 13:58:03Z bless $
%%

\chapter{Diskussion und Ausblick}
\label{ch:fazit}
%% ==============================

Eine Frage, die sich bei der Anwendung der Nemhauser-Trotter-Regel stellt ist, warum sie in der Praxis so niedrige Reduktionen erreicht. Zwar zeigte die Einzelanwendung Nemhauser-Trotter-Regel, dass die Ergebnisse im Vergleich zur Einzelanwendung der Kronenregel wie erwartet besser sind, allerdings war dies im weiteren Verlauf der Untersuchung nicht mehr zu beobachten. Hierfür könnten Probleminstanzen, bei denen die Nemhauser-Trotter-Regel gute Ergebnisse liefert genauer betrachtet werden, wodurch sich eventuell eine Modifikation, wie bei der Kronenregel ableiten lässt. Die Modifikation der Kronenregel gilt es weiterhin genauer zu untersuchen, da nicht garantiert ist, dass die in dieser Arbeit verwendete Einschränkung für das Finden des Matchings M$_{1}$ die besten Ergebnisse erzielt. Außerdem wurde nicht getestet, welchen Einfluss die Verwendung eines Maximum Matchings an dieser Stelle hätte. Des weiteren hat sich die Ähnlichkeiten bei der Reduktion zwischen Grad$_{1}$-Regel und Nemhauser-Trotter-Regel gezeigt. Bei der Kombination von Kronenregel und Grad$_{1}$-Regel führte die besagte Modifikation der Kronenregel dazu, dass verschiedene Kronen, beziehungsweise verschiedene Teile des Graphen betrachtet wurden und letztendlich zu einem drastisch besseren Ergebnis. Würde eine solche ebenfalls für die Nemhauser-Trotter-Regel werden, könnte diese in der Praxis besser anwendbar werden.
In Abbildung \ref{fig:trottCrownOne} zeigte sich, dass es scheinbar große Unterschiede in der Reduzierbarkeit der Graphen mit 3000 bis 4600 Knoten gibt. Hier wäre interessant festzustellen, welche Eigenschaften diese Gruppen unterscheiden und ob eventuell andere, in dieser Arbeit nicht verwendete Graphreduktionsregeln, wie die Folding-Regel erfolgreich anwendbar wären.\\
Insgesamt scheint es ein sehr großes Ungleichgewicht zwischen der Zeit, die für Preprocessing und für das Finden einer Knotenüberdeckung im Anschluss aufgebracht wird zu geben. Daher wäre es sinnvoll viel Aufwand bei der Vorverarbeitung zu betreiben, da wie in Kapitel \ref{ch:Einleitung:sec:Motivation} gezeigt, bereits eine einfache Verringerung von $k$ die Laufzeit eines Suchbaumalgorithmus' halbiert.

%%% Local Variables: 
%%% mode: latex
%%% TeX-master: "thesis"
%%% End: 