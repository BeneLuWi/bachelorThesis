%% zusammenf.tex
%% $Id: zusammenf.tex 61 2012-05-03 13:58:03Z bless $
%%

\chapter{Diskussion und Ausblick}
\label{ch:fazit}
%% ==============================

\begin{itemize}
\item Form von randomgraphen untersuchen
	\begin{itemize}
	\item Welche Form hinterlassen die Reduktionsregeln
	\item Welche Form wäre für die Nemhauser Trotter am besten
	\end{itemize}
\item Warum ist die Nemhauser-Trotter-Regel in der Praxis so schlecht?
\item Wie ergibt sich in bei den Reduktionen (Abbildung \ref{fig:trottCrownOne}) der Große Unterschied zwischen den Graphen im Bereich der Kantenmenge zwischen 3000 und 4600?
\item Wieso ist das Matching M1 so wichtig für die Kronenregel?
\item Implementierung der Grad 2 Regel
\item Insgesamt mehr Zeit auf Preprocessing verwenden, betrachtet man das Ungleichgewicht in der Laufzeit vom Finden einer Knotenüberdeckung im Vergleich zur Vorverarbeitung. Wenn die Ergebnisse der Vorverarbeitung vielversprechend sind, scheint es sinnvoll zumindest die Laufzeiten auszugleichen.
\end{itemize}

%%% Local Variables: 
%%% mode: latex
%%% TeX-master: "thesis"
%%% End: 
