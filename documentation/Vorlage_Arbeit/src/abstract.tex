\chapter*{Zusammenfassung}
%% ==============================
Gegenstand dieser Arbeit ist der Vergleich und die Anwendung von Graphreduktionsregeln für das Knotenüberdeckungsproblem. Die Regeln als solche, das Verhalten verschiedener Regeln in der Praxis und der Effekt von deren kombinierter Anwendung wurden untesucht. Umsetzung und Implementierung der Algorithmen geschah in der Programmiersprache \emph{C++} mithilfe der \emph{C++}-Library LEDA. Es stellte sich heraus, dass das Ergebnis der Reduktion im hohen Maße sowohl von der jeweiligen Kombination, als auch der Reihenfolge, in der die Reduktionsregeln an einem Problemgraphen angewandt werden, abhängt. Zudem wurde eine Konfiguration für die hier implementierte Kronenregel gefunden, welche die Reduktion der Regel in der Praxis deutlich verbessert. Die Abhängigkeit der Regel von dieser Einstellung wurde bereits durch andere Untersuchungen zum Thema entdeckt \cite{paper:7}. Die besten Ergebnisse bei der Reduktion pro Graph am eigens erstellten Testset erreichte die Anwendung in der Reihenfolge Nemhauser-Trotter-Regel - Kronenregel - Grad$_{1}$-Regel. Hier zeigten sich einige besondere Graphen, welche sich im Vergleich zu den anderen Graphen aus der gleichen Kategorie, beziehungsweise mit der gleichen Kantenanzahl, untypisch verhalten haben. 