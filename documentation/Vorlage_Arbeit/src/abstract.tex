\chapter*{Zusammenfassung}
%% ==============================
Gegenstand dieser Arbeit ist der Vergleich und die Anwendung von Graphreduktionsregeln für das Knotenübereckungsproblem. Zu untersuchen waren die Regeln als solche, das Verhalten verschiedener Regeln in der Praxis und der Effekt von deren kombinierter Anwendung. Für die Umsetzung und die Implementierung der Algorithmen wurden die Programmiersprache \emph{C++} und die \emph{C++}-Library LEDA verwendet. Es hat sich herausgestellt, dass das Ergebnis der Reduktion sowohl von der jeweiligen Kombination, als auch der Reihenfolge, in der die Reduktionsregeln an einem Problemgraphen angwandt werden einen starken Einfluss auf das Ergebnis haben. Zudem wurde eine Einstellung für die hier implementierte Kronenregel gefunden, welche die Reduktion der Regel deutlich verbessert. Diese Abhängigkeit wurde bereits durch andere Untersuchungen zum Thema entdeckt \cite{paper:7}. Die besten Ergebnisse bei der Reduktion am eigens erstellten Testset erreichten pro Graph die Anwendung in der Reihenfolge Nemhauser-Trotter-Regel - Kronenregel - Grad$_{1}$-Regel. Hier zeigten sich einige besondere Graphen, welche sich im Vergleich zu den anderen Graphen aus der gleichen Kategorie, beziehungsweise mit der gleichen Kantenanzahl, untypisch verhalten haben. 